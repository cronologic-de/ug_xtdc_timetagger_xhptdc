\section{Calibration of the 10G variant}

After performing a firmware update, the TimeTagger4-10G has to be re-calibrated.
This does \emph{only} apply to the 10G variant.

For this purpose, cronologic provides the \texttt{TT4\_10G\_Calibration\_64} command-line tool (located in the driver installation directory TODO).

To perform a calibration, you need a NIM signal with a constant frequency larger than 20\,kHz and smaller than 10\,MHz.

Run the \texttt{TT4\_10G\_Calibration\_64} tool and follow the instructions on-screen, that is:

\begin{itemize}
    \item Connect the NIM signal to channel Start and press enter. The tool will calibrate the Start channel.
    \item After Start was successfully calibrated, connect the same NIM signal to channel A and press enter. The tool will calibrate channel A.
    \item Repeat this process for channels B, C, and D, as well.
\end{itemize}

After a successful calibration of the TimeTagger4-10G, the message ``Calibrated all channels successfully'' will be displayed.
You can close the calibration tool.

In case the calibration fails, please check the following:
\begin{itemize}
    \item The TimeTagger4-10G is installed properly (see Section~\ref{sec:installation-of-board}).
    \item A proper NIM signal with a constant frequency larger than 20\,kHz and smaller than 10\,MHz is used.
    \item The NIM signal was connected to the appropriate input channel (see Figure~\ref{fig:bracket}).
\end{itemize}

If the calibration still fails, please contact \href{https://www.cronologic.de/contact}{cronologic support}.