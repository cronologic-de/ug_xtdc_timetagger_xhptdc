\txh{ %TimeTagger Version
    The \deviceName\ is a ``classic'' common start time-to-digital converter. 

    It records the time difference between a leading or trailing edge on the
    start input to the leading and trailing edges of the stop inputs.  Rising
    and falling edges of the stop channels A-D can be enabled individually. 
    The measurements are quantized as shown in Section~\ref{sec:features}. 
    The timestamps are recorded in integer multiples of the corresponding bin
    size.  Transitions of the input signals are called hits. To reliably
    detect hits the signal has to be stable for more than one quantization
    interval before and after the edge.  Triggers on the start channel must
    not occur less than \SI{5}{\nano\second} apart. The \deviceName\ records
    events without dead time at a readout rate of about 48\,MHits/s for Gen\,1
    and 60\,MHits/s for Gen\,2. For Gen\,2, the maximum readout rate
    of a single channel is 40\,MHits/s.
}{ %xTDC4
    The \deviceName\ is a ``classic'' common start time-to-digital converter. 

    It records the time difference between leading or trailing edges on the
    start input and the stop inputs.  Each stop channel A-D can be enabled
    individually.  The standard deviation of the timestamps is approx.
    8\,ps. The timestamps are recorded in integer multiples of
    a bin size of $5000/(3\times 128) = 13.0208\overline{3}$\,ps. The data
    acquisition can be limited to rising or falling signal transitions. 

    The maximum trigger rate on the start channel is \SI{4}{\mega\hertz}.

    \section{Handling of Difficult Hits}
    \label{difficulthits}
    Transitions of the input signals are called hits. To measure all hits with
    the maximum resolution the hits must fulfill all criteria set forth in
    this document.  However, the \deviceName{} does include mechanisms to
    provide as much information as possible for hits that fall out of these
    specs.

    To reliably detect hits the signal has to be stable for at least
    \SI{900}{\pico\second} before and after the edge that is to be measured. 
    Pulses as short as \SI{250}{\pico\second} are usually detected at full
    resolution but have a significant chance to be assigned to the wrong group
    or appear out of order.  For these hits bit 7 in the data word is set. See
    Section~\ref{flags} for more information on the data format.

    Between multiple hits on a stop channel a dead time of approximately
    \SI{5}{\nano\second} is required for full resolution.  Hits that are
    closer together will only be reported with a resolution of
    $5/6$\,ns = $833.\overline{3}$\,ps. For these
    hits both bits 6 and 7 are set.

    Data is copied from the 15-entry L0 FIFO to the larger downstream FIFOs at
    a rate of about \SI{12}{\mega\hertz} per channel.  If the L0 FIFO
    overflows the high resolution measurement of some hits will be discarded. 
    In this case a measurement from an alternative TDC will be used that has a
    resolution of about \SI{150}{\pico\second}.  For these hits bit 6 in the
    data word will be set.
} { %xHPTDC8 version
    The \deviceName\ is a streaming time-to-digital converter. It records the
    timestamps of changes at the inputs A-H in an infinite stream.  A flexible
    grouping mode is available that can emulate common-start or common-stop
    behavior. See Section~\ref{sec:grouping} for details.

    For each channel, it can be selected individually whether rising or
    falling edges are recorded. It is not possible to record both edges at the
    same channel.  The timestamps are recorded in integer multiples of a bin
    size of $5000/(3\times 128) = 13.0208\overline{3}$\,ps. 
    There must be at least \SI{5}{\nano\second} between consecutive hits on
    the same input channel to be detected reliably.  The \deviceName\ records
    events without dead time at a readout rate of about 48\,MHits/s. For a
    single channel, the maximum readout rate is about 12\,MHits/s.
}