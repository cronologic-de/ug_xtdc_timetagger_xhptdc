% NOTE: xHPTDC has a seperate file describing the readout

%%%%%%%%%%%%%%%%% run time control
\section{Run Time Control}

Once the devices are configured the following functions can be used to control
the behaviour of the devices.  All of these functions return quickly with very
little overhead, but they are not guaranteed to be thread safe.

\begin{description}[style=nextline]
    \item[\ttvar{int}{start\tu capture(}\device)]
    Start data acquisition.\par

    \item[\ttvar{int}{pause\tu capture(}\device)]
    Pause a started data acquisition. \par
    \texttt{pause} and \texttt{continue} have less overhead than start and
    stop but don't allow for a configuration change.

    \item[\ttvar{int}{continue\tu capture(}\device)]
    Call this to resume data acquisition after a call to
    \texttt{\prefix pause\tu capture()}. \par
    \texttt{pause} and \texttt{continue} have less overhead than start and
    stop but don't allow for a configuration change.

    \item[\ttvar{int}{stop\tu capture(}\device)]
    Stop data acquisition.\par
    
    \item[\protect{\parbox[b]{0.8\linewidth}{
        \ttvar{int}{start\tu tiger(}\device)\\
        \ttvar{int}{stop\tu tiger(}\device)}}]
    Start and stop the timing generator. This can be done independently of the
    state of the data acquisition.
\end{description}

%%%%%% readout
\section{Readout}
The device provides a stream of packets, that are read in batches. A batch of
packets is provided to the application, it processes them, by storing important information
in other structures. The batch that were processed need to be acknowledged, so that
the device can reuse the memory of these for the next data. That means processing
should be fast. 
\begin{lstlisting}[
    numbers=none,
]
timetagger4_read_in in;
// automatically acknowledge all data as processed
in.acknowledge_last_read = 1;
volatile crono_packet* p = read_data.first_packet;
timetagger4_read_out out;
int status = timetagger4_read(device, &in, &out);
if (status == CRONO_READ_OK) {
    while (p <= read_data.last_packet) {
        processPacket(p);
        p = crono_next_packet(p);
    }    
}
\end{lstlisting}
    
\begin{description}[style=nextline]
    \item[\ttvar{int}{acknowledge(}\device, \cronvar{crono\tu packet}{*packet)}]
    Acknowledges the processing of the last read block. This is only necessary
    if \texttt{\prefix read()} is not called with \texttt{in.acknowledge\tu
    last\tu read} set.\par
    This feature allows to either free up partial DMA space early if there
    will be no call to \texttt{\prefix read()} anytime soon.  It also allows
    to keep data over multiple calls to \texttt{\prefix read()} to avoid
    unnecessary copying of data.

    \item[\protect{\parbox[b]{\linewidth}{
        \ttvar{int}{read(}\device, \cronvar{\prefix read\tu in}{*in,} \\
        \hspace*{\labelwidth+\itemsep}\cronvar{\prefix read\tu out}{*out)}
    }}]
    Return a pointer to an array of captured data in \textsf{read\tu out}.
    The result contains a batch of packets of type \textsf{\prefix packet}.
    The batch is described by \texttt{first\tu packet} and \texttt{last\tu
    packet} in the \texttt{\prefix read\tu in} structure.
    
    \texttt{read\tu in} provides parameters to the driver. 
    A call to this method automatically allows the driver to reuse the memory
    returned in the previous call if \texttt{read\tu in.acknowledge\tu last\tu
    read} is set.\par
    Returns an error code as defined in the structure
    \texttt{\prefix read\tu out}.

    \item[\cronvar{crono\tu packet}{crono\tu next\tu packet(}\cronvar{crono\tu packet}{*packet})]
    Iterates to the next packet in the batch. 
\end{description}

\subsection{Input Structure \prefix read\tu in}
\begin{description}[style=nextline]
    \item[\cronvar{crono\tu bool\tu t}{acknowledge\tu last\tu read}]
    If set \texttt{\prefix read()} automatically acknowledges packets from the
    last read.  Otherwise \texttt{\prefix acknowledge()} needs to be called
    explicitly by the user. 
\end{description}

\subsection{Input Structure \prefix read\tu out}
\begin{description}[style=nextline]
    \item[\cronvar{crono\tu packet}{*first\tu packet}]
    Pointer to the first packet that was captured by the call of
    \texttt{\prefix read()}.

    \item[\cronvar{crono\tu packet}{*last\tu packet}]
    Address of header of the last packet in the buffer. This packet is still
    valid, all data after this packet is invalid.

    \item[\cronvar{int}{error\tu code}]
    Assignments of the error codes.\par
    \begin{tabular}{lc}
        \crondef{CRONO\tu READ\tu OK}                & \texttt{0}\\
        \crondef{CRONO\tu READ\tu NO\tu DATA}        & \texttt{1}\\
        \crondef{CRONO\tu READ\tu INTERNAL\tu ERROR} & \texttt{2}\\
        \crondef{CRONO\tu READ\tu TIMEOUT}           & \texttt{3}\\
    \end{tabular}

    \item[\cronvar{const char}{*error\tu message}]
    The last error in human readable form, possibly with additional
    information about the error.
\end{description}
    