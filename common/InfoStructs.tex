\section{Status Information}
	

\subsection{Functions for Information Retrieval}
The driver provides functions to retrieve detailed information on the
board type, its configuration, settings, and state.  The information is
split according to its scope and the computational requirements to query
the information from the board.

\ifxHPTDC{
    The information is provided on a per board basis. The parameter
    \texttt{index} selects which board is queried.
}{}

\begin{description}[style=nextline]
    \item[\ttvar{int}{get\tu device\tu type}(\device\ifxHPTDC{, \cronvar{int}{index}}{})]
    Returns the type of the device as \texttt{CRONO\tu DEVICE\tu
    \txh{TIMETAGGER4}{XTDC4}{XHPTDC8}}.

    \item[\ttvar{const char*}{get\tu last\tu error\tu
        message(\device\ifxHPTDC{, \cronvar{int}{index}}{})}]
    Returns most recent error message.
    \ifxHPTDC{
        If index is negative the last error message from the \\
        \texttt{\prefix manager} is returned.  Otherwise the last error message
        of the selected board is returned.
    }{}

    \item[\ttvar{int}{get\tu fast\tu info(}\deviceindex,
        \cronvar{\prefix fast\tu info}{*info)}]
    Returns fast dynamic information.\par
    This call gets a structure that contains dynamic information that can be
    obtained within a few microseconds.

    \item[\ttvar{int}{get\tu param\tu info(}\deviceindex, \cronvar{\prefix
        param\tu info}{*info)}]
    Returns configuration changes.\par
    Gets a structure that contains information that changes indirectly due to
    configuration changes.


    \item[\ttvar{int}{get\tu static\tu info(}\deviceindex, \cronvar{\prefix
        static\tu info}{*info)}]
    Contains static information.\par
    Gets a structure that contains information about the board that does not
    change during run time.

    \ifxHPTDC{
        \item[\protect{\parbox[b]{\linewidth}{
            \ttvar{int}{get\tu temperature\tu info(}\deviceindex,\\
            \cronvar{\prefix temperature\tu info}{*info)}}}]
        Get temperature measurements from multiple sources on the board.

        \item[\ttvar{int}{get\tu clock\tu info(}\deviceindex, \cronvar{\prefix
            clock\tu info}{*info)}]
        Get information on clocking configuration and status.

        \item[\ttvar{const char*}{device\tu state\tu to \tu
            str(}\cronvar{int}{state)}]
        Convert the state value from \texttt{\prefix fast\tu info.state} into
        a human-readable string. 
    }{} 
\end{description}

%%%%%%%%%%%%%%%%% static info

\subsection{Structure \prefix static\tu info}

This structure contains information about the board that does not change during
run time. It is provided by the function
\texttt{\prefix get\tu static\tu info()}.

\begin{description}[style=nextline]
    \item[\cronvar{int}{size}]
    The number of bytes occupied by the structure.

    \item[\cronvar{int}{version}]
    A version number that is increased when the definition of the structure is
    changed. The increment can be larger than one to match driver version
    numbers or similar.

    \item[\cronvar{int}{board\tu id}]
    ID of the board.\par
    \ifxHPTDC{
        All \deviceName\ boards in the system are numbered in the order of
        their serial numbers starting at zero.  Channel A of a board has
        channel number $index \times 10$.
    }{}

    \item[\cronvar{int}{driver\tu revision}]
    Encoded version number for the driver.\par
    The lower three bytes contain a triple-level hierarchy of version numbers,
    e.g., \texttt{0x010103} encodes version 1.1.3.\par
    The version adheres to the Semantic Versioning scheme as defined at
    \href{https://semver.org}{https://semver.org}. A change in the first digit
    generally requires a recompilation of user applications.  Changes in the
    second digit denote significant improvements or changes that don't break
    compatibility and the third digit increments with minor bug fixes and
    similar updates that do not affect the API.

    \item[\cronvar{int}{driver\tu build\tu revision}]
    Build number of the driver according to cronologic's internal versioning
    system.

    \item[\cronvar{int}{firmware\tu revision}]
    Revision number of the FPGA configuration.

    \item[\cronvar{int}{board\tu revision}]
    Board revision number.\par
    The board revision number can be read from a register. It is a four-bit
    number that changes when the schematic of the board is changed. This should
    match the revision number printed on the board.

    \item[\cronvar{int}{board\tu configuration}]
    Describes the schematic configuration of the board.\par
    The same board schematic can be populated in multiple variants. This is an
    8-bit code that can be read from a register.

    \item[\cronvar{int}{subversion\tu revision}]
    Subversion revision id of the FPGA configuration source code.

    \txh{
        \item[\cronvar{int}{chip\tu id}]
        Reserved.
    }{
        \item[\cronvar{int}{chip\tu id}]
        16 bit factory ID of the TDC chip.
    }{
        \item[\cronvar{int}{chip\tu id[2]}]
        16 bit factory ID for each of the TDC chips.
    }

    \item[\cronvar{int}{board\tu serial}]
    Serial number.\par
    Year and running number are concatenated in 8.24 format. The number is
    identical to the one printed on the silvery sticker on the board.

    \item[\protect{\parbox[b]{0.8\linewidth}{
        \cronvar{\uint}{flash\tu serial\tu high}\\
        \cronvar{\uint}{flash\tu serial\tu low}}}]
    64-bit manufacturer serial number of the flash chip

    \item[\cronvar{crono\tu bool\tu t}{flash\tu valid}]
    If not \texttt{0}, the driver found valid calibration data in the flash on
    the board and is using it. This value is not applicable for the
    \deviceName.

    \item[\cronvar{char}{calibration\tu date[20]}]
    DIN EN ISO 8601 string YYYY-MM-DD HH:MM of the time when the card
    was calibrated.

    \item[\cronvar{char}{bitstream\tu date[20]}]
    DIN EN ISO 8601 string YYYY-MM-DD HH:MM of the time when the bitstream on
    the card was created.

    \txh{
        \item[\cronvar{double}{delay\tu bin\tu size}]
        Bin size of delay in picoseconds. The increment of the
        \texttt{delay\tu config.delay} field for the \deviceName.

        \item[\cronvar{double}{auto\tu trigger\tu ref\tu clock}]
        Auto trigger clock frequency. The clock frequency of the auto trigger
        in Hertz used for calculating the \texttt{auto\tu trigger\tu period}.

        \item[\cronvar{uint32\tu t}{rollover\tu period}]
        The number of bins in a rollover period. This is a power of 2 (the
        maximum value of a hit timestamp is this value minus 1)
    }{}{}
\end{description}

%%%%%%%%%%%%%%%%%%%%%%%% param info
\subsection{Structure \prefix param\tu info}
This structure contains configuration changes provided by \texttt{\prefix
get\tu param\tu info()}.

\begin{description}[style=nextline]
    \item[\cronvar{int}{size}]
    The number of bytes occupied by the structure.

    \item[\cronvar{int}{version}]
    A version number that is increased when the definition of the structure is
    changed. The increment can be larger than one to match driver version
    numbers or similar.

    \item[\cronvar{double}{binsize}]
    Bin size (in \si{\pico\second}) of the measured TDC data.

    \ifxHPTDC{}{ %board ID is found in the static_info structure for .
        \item[\cronvar{int}{board\tu id}]
        Board ID.\par
        The board uses this ID to identify itself in the output data stream.
        The ID takes values between \texttt{0} and \texttt{255}.
    }

    \item[\cronvar{int}{channels}]
    Number of TDC channels of the board.\par
    Fixed at \texttt{\ifxHPTDC{8}{4}}.

    \item[\cronvar{int}{channel\tu mask}]
    Bit assignment of each enabled input channel.\par
    Bit $\texttt{0} \leq n < \texttt{\ifxHPTDC{8}{4}}$ is set if channel $n$ is
    enabled.

    \item[\cronvar{int64\tu t}{total\tu buffer}]
    The total amount of DMA buffer in bytes.

    \ifxHPTDC{}{
        \item[\cronvar{double}{packet\tu binsize}]
        \itett{ %tt4
            For the \deviceName\, the packet binsize is equal to the binsize
            and depends on the generation of the card. Gen\,1 boards have a
            packet binsize of \SI{500}{\pico\second}, while Gen\,2 boards have
            \SI{100}{\pico\second}.
        }{ %xtdc4
            For \deviceName\, this is \SI{1666.6}{\pico\second}
        }
    }

    \ifxHPTDC{}{
        \item[\cronvar{double}{quantisation}]
        \itett{ %tt4
            Quantisation or measurement resolution. Depending on the board
            variant this ranges from \SIrange{100}{1000}{\pico\second}.\par
            \begin{tabular} {|l|l|l|l|l|l|l}
            \hline
            --1G & --2G & --1.25G & --2.5G & --5G & --10G \\
            \hline
            1000 ps & 500 ps & 800 ps & 400 ps & 200 ps & 100 ps\\
            \hline
            \end{tabular}

            This means, that for --1.25G the lower 3 bits in the timestamps are
            zero.
        }{ %xtdc4
            Quantisation or measurement resolution. For the \deviceName\, this
            is \SI{13.0208}{\pico\second}
        }
    }
\end{description}

%%%%%%%%%%%%%%%%%%%%%%%%%% fast info

\subsection{Structure \prefix fast\tu info} \label{structfastinfo}
\begin{description}[style=nextline]
    \item[\cronvar{int}{size}]
    The number of bytes occupied by the structure. 

    \item[\cronvar{int}{version}]
    A version number that is increased when the definition of the structure is
    changed.  The increment can be larger than one to match driver version
    numbers or similar.

    \ifxHPTDC{} {
        \item[\cronvar{int}{tdc\tu rpm}]
        Speed of the TDC fan in rounds per minute. Reports \texttt{0} if no fan
        is present.
    }

    \item[\cronvar{int}{fpga\tu rpm}]
    Speed of the FPGA fan in rounds per minute. Reports \texttt{0} if no fan is
    present.

    \item[\cronvar{int}{alerts}]
    Alert bits from the temperature sensor and the system monitor.
    \itett{
        The TimeTagger4 does not implement any temperature alerts.
    }{
        Bit 0 is set if the TDC temperature exceeds \SI{140}{\degreeCelsius}. 
        In this case the TDC did shut down and the device needs to be
        reinitialized. 
    }

    \item[\cronvar{int}{pcie\tu pwr\tu mgmt}]
    Always \texttt{0}.

    \item[\cronvar{int}{pcie\tu link\tu width}]
    Number of PCIe lanes the card uses. Should always be \texttt{10} for the
    \deviceName.

    \item[\cronvar{int}{pcie\tu max\tu payload}]
    Maximum size in bytes for one PCIe transaction. Depends on system
    configuration.

    \ifxHPTDC{
    \item[\cronvar{int}{state}]
    Current state of the \deviceName. 

    \begin{tabular}{lc}
        \cronvar{const static int}{\PREFIX DEVICE\tu STATE\tu CREATED} & \texttt{0}  \\*
        \cronvar{const static int}{\PREFIX DEVICE\tu STATE\tu INITIALIZED} & \texttt{1}  \\*
        \cronvar{const static int}{\PREFIX DEVICE\tu STATE\tu CONFIGURED} & \texttt{2}  \\*
        \cronvar{const static int}{\PREFIX DEVICE\tu STATE\tu CAPTURING} & \texttt{3}  \\*
        \cronvar{const static int}{\PREFIX DEVICE\tu STATE\tu PAUSED} & \texttt{4}  \\*
        \cronvar{const static int}{\PREFIX DEVICE\tu STATE\tu CLOSED} & \texttt{5}  \\*
    \end{tabular}
    }{}
\end{description}
%%%%%%%%%%%%%%%%%%%%%%% temperature info

\ifxHPTDC{
    \subsection{Structure \prefix temperature\tu info}
    This structure provide the values of temperature measurements of
    various chips on the board.

    \begin{description}[style=nextline]
        \item[\cronvar{int}{size}]
        The number of bytes occupied by the structure.

        \item[\cronvar{int}{version}]
        A version number that is increased when the definition of the
        structure is changed. The increment can be larger than one to match
        driver version numbers or similar.

        \item[\cronvar{float}{tdc[2]}]
        Temperature for each of the TDC chips in \si{\degreeCelsius}.
    \end{description}

    %%%%%%%%%%%%%%%%%%%%% clock info

    \subsection{Structure \prefix clock\tu info}
    This structure provides information about the clock network of the
    device. 

    \begin{description}[style=nextline]
        \item[\cronvar{int}{size}]
        The number of bytes occupied by the structure.

        \item[\cronvar{int}{version}]
        A version number that is increased when the definition of the structure
        is changed. The increment can be larger than one to match driver
        version numbers or similar.\par

        \item[\cronvar{crono\tu bool\tu t}{cdce\tu locked}]
        Set if the jitter cleaning PLL clock synthesizer achieved lock.

        \item[\cronvar{int}{cdce\tu version}]
        Version information from the CDCE62005 clock synthesizer.
        
        \item[\cronvar{crono\tu bool\tu t}{use\tu ext\tu clock}]
        Source for the clock synthesizer is usually the 10~MHz onboard
        oscillator. During initialization, alternatively an external clock
        on J2 can be selected.  When multiple boards are synchronized all board
        use a common external clock. See section \ref{multiboard} for details.
    
        \item[\cronvar{crono\tu bool\tu t}{fpga\tu locked}]
        Set if the FPGA datapath PLL achieved lock.
    \end{description}
}{}