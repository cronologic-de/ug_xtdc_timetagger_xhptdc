\section{Status Information}
	

\subsection{Functions for Information Retrieval}
    The driver provides functions to retrieve detailed information on the board type, its configuration, settings, and state. 
    The information is split according to its scope and the computational requirements to query the information from the board.
    
    \ifxHPTDC{
        The information is provided on a per board basis. The parameter \textsf{index} selects which board is queried.
    }{}

    \ttvar{int}{get\tu device\tu type}(\device\ifxHPTDC{,\lb \cronvar{int}{index}}{})\\
    Returns the type of the device as \textsf{CRONO\tu DEVICE\tu \txh{TIMETAGGER4}{XTDC4}{XHPTDC8}}.\par

    \ttvar{const char*}{get\tu last\tu error\tu message(\device\ifxHPTDC{,\lb \cronvar{int}{index}}{})}\\
    Returns most recent error message.
    \ifxHPTDC{If index is negative the last error message from the \textsf{\prefix manager} is returned. 
    Otherwise the last error message of the selected board is returned. }{}
    \par

    \ttvar{int}{get\tu fast\tu info(}\deviceindex, \lb\cronvar{\prefix fast\tu info}{*info)}\\
    Returns fast dynamic information.\\
    This call gets a structure that contains dynamic information that can be obtained within a few microseconds.\par

    \ttvar{int}{get\tu param\tu info(}\deviceindex, \lb\cronvar{\prefix param\tu info}{*info)}\\
    Returns configuration changes.\\
    Gets a structure that contains information that changes indirectly due to configuration changes.\par


    \ttvar{int}{get\tu static\tu info(}\deviceindex, \lb\cronvar{\prefix static\tu info}{*info)}\\
    Contains static information.\\
    Gets a structure that contains information about the board that does not change during run time.\par 

   \ifxHPTDC{
        \ttvar{int}{get\tu temperature\tu info(}\deviceindex, \lb\cronvar{\prefix temperature\tu info}{*info)}\\
        Get temperature measurements from multiple sources on the board.

        \ttvar{int}{get\tu clock\tu info(}\deviceindex, \lb\cronvar{\prefix clock\tu info}{*info)}\\
        Get information on clocking configuration and status.

        \ttvar{const char *}{device\tu state\tu to \tu str(}\cronvar{int state)}\\
        Convert the state value from \textsf{\prefix fast\tu info.state} into a human readable string. 
   }{   
        \ttvar{int}{get\tu slow\tu info(}\deviceindex, \lb\cronvar{\prefix slow\tu info}{*info)}\\
        Returns slow dynamic information.\\
        The data reported in this structure requires milliseconds to be obtained. 
        The application should only call it in situations where the program flow can be blocked as long.\par
   } 

%%%%%%%%%%%%%%%%% static info

\subsection{Structure \prefix static\tu info}

This structure contains information about the board that does not change during run time. It is provided by the function \textsf{\prefix get\tu static\tu info()}.\par

\cronvar{int}{size}\\
The number of bytes occupied by the structure.

\cronvar{int}{version}\\
A version number that is increased when the definition of the structure is changed. The increment can be larger than one to match driver version numbers or similar.\par

\cronvar{int}{board\tu id}\\
ID of the board.\\
\ifxHPTDC{
    All \deviceName\ boards in the system are numbered in the order of their serial numbers starting at zero.
    Channel A of a board has channel number $index \cdot 10$. \par
}{}

\cronvar{int}{driver\tu revision}\\
Encoded version number for the driver.\\
The lower three bytes contain a triple-level hierarchy of version numbers, e.g. 0x010103 encodes version 1.1.3.\\
The version adheres to the Semantic Versioning scheme as defined at \href{https://semver.org}{https://semver.org}. A change in the first digit generally requires a recompilation of user applications. 
Changes in the second digit denote significant improvements or changes that don't break compatibility 
and the third digit increments with minor bug fixes and similar updates that do not affect the API.\par

\cronvar{int}{driver\tu build\tu revision}\\
Build number of the driver according to cronologic's internal versioning system.

\cronvar{int}{firmware\tu revision}\\
Revision number of the FPGA configuration.

\cronvar{int}{board\tu revision}\\
Board revision number.\\
The board revision number can be read from a register. It is a four-bit number that changes when the schematic of the board is changed. This should match the revision number printed on the board.

\cronvar{int}{board\tu configuration}\\
Describes the schematic configuration of the board.\\
The same board schematic can be populated in multiple variants. This is an eight-bit code that can be read from a register.

\cronvar{int}{subversion\tu revision}\\
Subversion revision id of the FPGA configuration source code.

\txh{
    \cronvar{int}{chip\tu id}\\
    Reserved.
}{
    \cronvar{int}{chip\tu id}\\
    16 bit factory ID of the TDC chip.
}{
    \cronvar{int}{chip\tu id[2]}\\
    16 bit factory ID for each of the TDC chips.
}\par

\cronvar{int}{board\tu serial}\\
Serial number.\\
Year and running number are concatenated in 8.24 format. The number is identical to the one printed on the silvery sticker on the board.\par

\cronvar{\uint}{flash\tu serial\tu high}\\
\cronvar{\uint}{flash\tu serial\tu low}\\
64-bit manufacturer serial number of the flash chip

\cronvar{crono\tu bool\tu t}{flash\tu valid}\\
If not 0 the driver found valid calibration data in the flash on the board and is using it.\par

\cronvar{char}{calibration\tu date[20]}\\
DIN EN ISO 8601 string YYYY-MM-DD HH:DD describing the time when the card was calibrated.

%%%%%%%%%%%%%%%%%%%%%%%% param info

\subsection{Structure \prefix param\tu info}
This structure contains configuration changes provided by \textsf{\prefix get\tu param\tu info()}.

\cronvar{int}{size}\\
The number of bytes occupied by the structure. \par

\cronvar{int}{version}\\
A version number that is increased when the definition of the structure is changed. The increment can be larger than one to match driver version numbers or similar.\par


\cronvar{double}{binsize}\\
Bin size (in ps) of the measured TDC data.

\ifxHPTDC{}{ %board ID is found in the static_info structure for .
    \cronvar{int}{board\tu id}\\
    Board ID.\\
    The board uses this ID to identify itself in the output data stream. The ID takes values between 0 and 255.\par
}

\cronvar{int}{channels}\\
Number of TDC channels of the board.\\
Fixed at \ifxHPTDC{8}{4}.\par

\cronvar{int}{channel\tu mask}\\
Bit assignment of each enabled input channel.\\
Bit $0 \leq n < \ifxHPTDC{8}{4}$ is set if channel $n$ is enabled. \par

\cronvar{\longlong}{total\tu buffer}\\
The total amount of DMA buffer in bytes.

%%%%%%%%%%%%%%%%%%%%%%%%%% fast info

\subsection{Structure \prefix fast\tu info}
\label{structfastinfo}
\cronvar{int}{size}\\
The number of bytes occupied by the structure. \par

\cronvar{int}{version}\\
A version number that is increased when the definition of the structure is changed. 
The increment can be larger than one to match driver version numbers or similar.\par

\ifxHPTDC{} {
    \cronvar{int}{tdc\tu rpm}\\
    Speed of the TDC fan in rounds per minute. Reports 0 if no fan is present.\par
}
\cronvar{int}{fpga\tu rpm}\\
Speed of the FPGA fan in rounds per minute. Reports 0 if no fan is present.\par

\cronvar{int}{alerts}\\
Alert bits from the temperature sensor and the system monitor.
\itett{
    The TimeTagger4 does not implement any temperature alerts.
}{
    Bit 0 is set if the TDC temperature exceeds 140°C. 
    In this case the TDC did shut down and the device needs to be reinitialized. 
}\par

\cronvar{int}{pcie\tu pwr\tu mgmt}\\
Always 0. \par

\cronvar{int}{pcie\tu link\tu width}\\
Number of PCIe lanes the card uses. Should always be 1 for the \deviceName. \par

\cronvar{int}{pcie\tu max\tu payload}\\
Maximum size in bytes for one PCIe transaction. Depends on system configuration.\par

\cronvar{int}{state}\\
The state the \deviceName\ is in.

\begin{tabular}{lc}
    \cronvar{const static int}{\PREFIX DEVICE\tu STATE\tu CREATED} & 0  \\*
    \cronvar{const static int}{\PREFIX DEVICE\tu STATE\tu INITIALIZED} & 1  \\*
    \cronvar{const static int}{\PREFIX DEVICE\tu STATE\tu CONFIGURED} & 2  \\*
    \cronvar{const static int}{\PREFIX DEVICE\tu STATE\tu CAPTURING} & 3  \\*
    \cronvar{const static int}{\PREFIX DEVICE\tu STATE\tu PAUSED} & 4  \\*
    \cronvar{const static int}{\PREFIX DEVICE\tu STATE\tu CLOSED} & 5  \\*
\end{tabular}\par

%%%%%%%%%%%%%%%%%%%%%%% temperature info

\ifxHPTDC{
    \subsection{Structure \prefix temperature\tu info}
        This structure provide the values of temperature measurements of various chips on the board.

        \cronvar{int}{size}\\
        The number of bytes occupied by the structure. \par

        \cronvar{int}{version}\\
        A version number that is increased when the definition of the structure is changed. The increment can be larger than one to match driver version numbers or similar.\par

        \cronvar{float}{tdc[2]}\\
        Temperature for each of the TDC chips in °C

    %%%%%%%%%%%%%%%%%%%%% clock info

    \subsection{Structure \prefix clock\tu info}
        This structure provides information about the clock network of the device. 

        \cronvar{int}{size}\\
        The number of bytes occupied by the structure. \par

        \cronvar{int}{version}\\
        A version number that is increased when the definition of the structure is changed. The increment can be larger than one to match driver version numbers or similar.\par

        \cronvar{crono\tu bool\tu t}{cdce\tu locked}\\
        Set if the jitter cleaning PLL clock synthesizer achieved lock.

        \cronvar{int}{cdce\tu version}\\
        Version information from the CDCE62005 clock synthesizer.\\
        
        \cronvar{crono\tu bool\tu t}{use\tu ext\tu clock}\\
        Source for the clock synthesizer is usually the 10~MHz onboard oscillator. During initialization, alternatively an external clock on J2 can be selected.
        When multiple boards are synchonized all board use a common external clock. See section \ref{multiboard} for details.
        \\       

    
        \cronvar{crono\tu bool\tu t}{fpga\tu locked}\\
        Set if the FPGA datapath PLL achieved lock.\\
    
}{}