\subsection{Structure \prefix static\tu info}

This structure contains information about the board that does not change during run time. It is provided by the function \textsf{\prefix get\tu static\tu info}.\par

\cronvar{int}{size}\\
The number of bytes occupied by the structure.

\cronvar{int}{version}\\
A version number that is increased when the definition of the structure is changed. The increment can be larger than one to match driver version numbers or similar. Currently only version 0 is defined.\par


\cronvar{int}{board\tu id}\\
ID of the board.\\
This value has been passed to the constructor by the user. It is reflected in the output data.\par

\cronvar{int}{driver\tu revision}\\
Encoded version number for the driver.\\
The lower three bytes contain a triple level hierarchy of version numbers, e.g. 0x010103 encodes version 1.1.3.\\
A change in the first digit generally requires a recompilation of user applications. 
Changes in the second digit denote significant improvements or changes that don't break compatibility 
and the third digit increments with minor bug fixes and similar updates that do not affect the API.\par

\cronvar{int}{firmware\tu revision}\\
Revision number of the FPGA configuration.

\cronvar{int}{board\tu revision}\\
Board revision number.\\
The board revision number can be read from a register. It is a four-bit number that changes when the schematic of the board is changed. This should match the revision number printed on the board.

\cronvar{int}{board\tu configuration}\\
Describes the schematic configuration of the board.\\
The same board schematic can be populated in multiple variants. This is a four bit-code that can be read from a register.

\cronvar{int}{subversion\tu revision}\\
Subversion revision id of the FPGA configuration source code.

\cronvar{int}{chip\tu id}\\
\itett{
    Reserved.
    }{
    16 bit factory ID of the TDC chip.
    }\par

\cronvar{int}{board\tu serial}\\
Serial number.\\
With year and running number in 8.24 format. The number is identical to the one printed on the silvery sticker on the board.\par

\cronvar{unsigned int}{flash\tu serial\tu high}\\
high 32 bits of the 64-bit manufacturer serial number of the flash chip.

\cronvar{unsigned int}{flash\tu serial\tu low}\\
low 32 bits of the 64-bit manufacturer serial number of the flash chip

\cronvar{int}{flash\tu valid}\\
If not 0 the driver found valid calibration data in the flash on the board and is using it.\par

\subsection{Structure \prefix param\tu info}
This struct contains configuration changes provided by \textsf{\prefix get\tu param\tu info()}.

\cronvar{int}{size}\\
The number of bytes occupied by the structure. \par

\cronvar{int}{version}\\
A version number that is increased when the definition of the structure is changed. The increment can be larger than one to match driver version numbers or similar. Currently only version 0 is defined.\par


\cronvar{double}{binsize}
Bin size (in ps) of the measured TDC data.

\cronvar{int}{board\tu id}\\
Board ID.\\
The board uses this ID to identify itself in the output data stream. The ID takes values between 0 and 255.\par

\cronvar{int}{channels}\\
Number of channels of the board.\\
Returns 4.\par

\cronvar{int}{channel\tu mask}\\
Bit assignment of each enabled input channel.\\
Bit $n <= 0 < 4$ is set if channel n is enabled. \par

\cronvar{\tu\tu int64}{total\tu buffer}\\
The total amount of DMA buffer in bytes.

\subsection{Structure \prefix fast\tu info}

\cronvar{int}{size}\\
The number of bytes occupied by the structure. \par

\cronvar{int}{version}\\
A version number that is increased when the definition of the structure is changed. The increment can be larger than one to match driver version numbers or similar. Currently only version 0 is defined.\par


\cronvar{int}{tdc\tu rpm}\\
Speed of the TDC fan in rounds per minute. Reports 0 if no fan is present.\par

\cronvar{int}{fpga\tu rpm}\\
Speed of the FPGA fan in rounds per minute. Reports 0 if no fan is present.\par

\cronvar{int}{alerts}\\
Alert bits from temperature sensor and the system monitor.
\itett{
    The TimeTagger4 does not implement any temperature alerts.
}{
    Bit 0 is set if the TDC temperature exceeds 140°C. The TDC did shut down and the device needs to be reinitialized. 
}\par

\cronvar{int}{pcie\tu pwr\tu mgmt}
Allways 0. \par

\cronvar{int}{pcie\tu link\tu width}\\
Number of PCIe lanes the card uses. Should always be 1 for the \deviceName. \par

\cronvar{int}{pcie\tu max\tu payload}\\
Maximum size in bytes for one PCIe transaction. Depends on system configuration.\par