For each measured edge, the \deviceName\ creates a 12-byte data structure
TDCHit that contains a 64-bit timestamp in picoseconds and three fields with
additional information. 

\section{Structure TDCHit}
\label{TDCHit}
\begin{description}[style=nextline]
    \item[\cronvar{int64\_t}{time}]
    The timestamp of the hit in picoseconds. 

    If grouping is disabled the timestamps are continuously counting up from
    the call to \texttt{\prefix start\tu capture()}.

    If grouping is enabled the timestamps are relative to the trigger or the
    separate zero reference of the group.  The first TDCHit of a group has
    channel number 255 and provides the absolute time of the group.  The
    absolute time of each of the hits can be obtained by adding this value to
    each of the relative timestamps.

    \item[\cronvar{uint8\_t}{channel}]
    For the first board in the system, this is \texttt{0} to \texttt{7} for the TDC channels A
    to H, or \texttt{8} to \texttt{9} for ADC data. Data from channels 8 and 9 should usually be
    treated as data from the same channel.  For the other boards, the channel
    number is incremented by $\texttt{board\_id} \times 10$.  In grouping
    mode, the first hit of each group has channel number 255 and contains the
    absolute time of the group.

    \item[\cronvar{uint8\_t}{type}]
    Additional information on the type of hit recorded
    (see~Section~\ref{hittypes}).

    \item[\cronvar{uint16\_t}{bin}]
    For ADC hits this contains the sampled voltage. For TDC hits the content
    is undefined.
\end{description}

\subsection{TDCHit Types \label{hittypes}}
\newcommand{\HTYPE}{\PREFIX TDCHIT\tu TYPE\tu}

\subsubsection{Type information for TDC measurements}
If the hit is a TDC measurement on channels A to H the following flags are
defined for the \texttt{type} field of the TDCHit structure:\\*
\begin{description}[style=nextline]
    \item[\crondef{\HTYPE RISING} 0x01]
    Rising edge
    \item[\crondef{\HTYPE ERROR} 0x02]
    any type of error
    \item[\crondef{\HTYPE ERROR\tu TIMESTAMP\tu LOST} 0x04]
    Hits missing due to L1 FIFO overflow
    \item[\crondef{\HTYPE ERROR\tu ROLLOVER\tu LOST} 0x08]
    Invalid timestamp due to internal error
    \item[\crondef{\HTYPE ERROR\tu PACKETS\tu LOST} 0x10]
    Hits missing due to a lost DMA packet
    \item[\crondef{\HTYPE ERROR\tu SHORTENED} 0x20]
    Hits missing due to a shortened DMA packet
    \item[\crondef{\HTYPE ERROR\tu DMA\tu FIFO\tu FULL} 0x40]
    Hits missing due to L2 FIFO overflow
    \item[\crondef{\HTYPE ERROR\tu HOST\tu BUFFER\tu FULL} 0x80]
    Hits missing due to host buffer overflow
\end{description}

If hits are missing the error flag is set on the next hit from the same board
that is read out.

\subsubsection{Type information for ADC measurements}
If the hit is an ADC measurement on channels 8 or 9, the following flags are
defined for the \texttt{type} field of the TDCHit structure:
\begin{description}[style=nextline]
    \item[\crondef{\HTYPE ADC\tu INTERNAL} 0x01]
    \indent ADC measurement triggered by internal strobe
    \item[\crondef{\HTYPE ADC\tu ERROR} 0x02]
    \indent any type of error
    \item[\crondef{\HTYPE ADC\tu ERROR\tu INVALID\tu TRIGGER} 0x08]
    \indent TRG input violated timing requirements. Data may be corrupted
    \item[\crondef{\HTYPE ADC\tu ERROR\tu DATA\tu LOST} 0x10]
    \indent ADC measurement missing due to overflow of any buffer
\end{description}

If hits are missing the error flag is set on the next hit from the same board
that is read out.


