	%%%%%%%%%%%%%%%%% run time control
	\section{Controlling the State of the Driver}
		\ttvar{int}{start\tu capture(}\device)\\
		Start data acquisition.\par

		\ttvar{int}{pause\tu capture(}\device)\\
		Pause a started data acquisition. 
		Pause and continue have less overhead than start and stop but don't allow for a configuration change.\par

		\ttvar{int}{continue\tu capture(}\device)\\
		Call this to resume data acquisition after a call to \textsf{\prefix pause\tu capture()}.
		Pause and continue have less overhead than start and stop but don't allow for a configuration change.\par

		\ttvar{int}{stop\tu capture(}\device)\\
		Stop data acquisition.\par

		\ttvar{int}{start\tu tiger(}\device, \cronvar{int}{index})\\
		\ttvar{int}{stop\tu tiger(}\device, \cronvar{int}{index})\\
		Start and stop the timing generator of an individual board. 
		This can be done independently of the state of the data acquisition.\par	


\section{Readout}

\ttvar{int}{read\tu hits(}\device,  \cronvar{TDCHit}{*hit\tu buf,} \cronvar{size\tu t}{size} )\\
Read a multitude of hits into a buffer provided by the user. Returns the number of read hits.\\
If grouping is enabled a single group is read. 
If the group is to large for the buffer the remaining hits of the group are discarded.\\*
If grouping is disabled, all availabe data is read up to the size of the buffer. \\
The method always returns immediately. If no hits are read it of is beneficial to call \textsf{sleep()} 
or yield the CPU to another process instead of trying again immediately.\\
Make sure to set \textsf{size} to the number of elemenst that fit into the buffer.\\*




