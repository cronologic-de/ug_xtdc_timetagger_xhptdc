%%%%%%%%%%%%%%%%% run time control
\section{Controlling the State of the Driver}
Once the devices are configured the following functions can be used to control
the behaviour of the devices.  All of these functions return quickly with very
little overhead, but they are not guaranteed to be thread safe.

\begin{description}[style=nextline]
    \item[\ttvar{int}{start\tu capture(})]
    Start data acquisition.

    \item[\ttvar{int}{pause\tu capture(})]
    Pause a started data acquisition. 
    Pause and continue have less overhead than start and stop but don't allow
    for a configuration change.

    \item[\ttvar{int}{continue\tu capture(})]
    Call this to resume data acquisition after a call to
    \texttt{\prefix pause\tu capture()}.  Pause and continue have less overhead
    than start and stop but don't allow for a configuration change.

    \item[\ttvar{int}{stop\tu capture(})]
    Stop data acquisition.

    \item[\protect{\parbox[b]{\linewidth}{
    \ttvar{int}{start\tu tiger(}\cronvar{int}{index})\\
    \ttvar{int}{stop\tu tiger(}\cronvar{int}{index})}}]
    Start and stop the timing generator of an individual board. 
    This can be done independently of the state of the data acquisition.
    
    \item[\ttvar{int}{software\tu trigger(}\cronvar{int}{index})]
    Sets the software trigger for one clock cycle.  This can be configured for
    the TiGer and for the gating blocks as trigger source
    \texttt{\PREFIX TRIGGER\tu SOURCE\tu SOFTWARE}. 
\end{description}

\section{Readout}
\begin{description}[style=nextline]
    \item[\ttvar{int}{read\tu hits(}\cronvar{TDCHit}{*hit\tu buf,} \cronvar{size\tu t}{size})]
    Read a multitude of hits into a buffer provided by the user. Returns the
    number of read hits.\par
    If grouping is enabled a single group is read.  If the group is too large
    for the buffer the remaining hits of the group are discarded.\par
    If grouping is disabled, all available data is read up to the size of the
    buffer. \par
    The method always returns immediately. If no hits are read, it is
    beneficial to call \texttt{sleep()} or yield the CPU to another process
    instead of trying again immediately.\par
    Make sure to set \texttt{size} to the number of elements that fit into the
    buffer.\par

    This function is not thread-safe.  If you want to process the read data in
    multiple threads the data needs to be copied to a separate buffer for each
    thread.

    \ifxHPTDC{
        \item[\ttvar{int}{get\tu current\tu timestamp(}\cronvar{int}{index,} \cronvar{int64\_t}{*timestamp})]
        Return current internal timestamp counter value of the selected 
        \deviceName{} in picoseconds.
    }{}
\end{description}
